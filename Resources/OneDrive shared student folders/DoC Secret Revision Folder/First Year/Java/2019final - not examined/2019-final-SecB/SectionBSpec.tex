\documentclass[a4wide,11pt]{article}

\usepackage{alltt}
\usepackage{amsmath}
\usepackage{color}
\usepackage{comment}
\usepackage{graphicx}
\usepackage{lscape}
\usepackage{multicol}
\usepackage{parskip}
\usepackage{url}
\usepackage{verbatim}
\usepackage{wrapfig}
\usepackage{tabularx}
\usepackage{float}

\usepackage[a4paper]{geometry}
\geometry{top=1.0in, bottom=1.0in, left=1.0in, right=1.0in}

\renewcommand{\j}[1]{\texttt{#1}}

\newcommand{\sumop}{\mathsf{sum}}
\newcommand{\prodop}{\mathsf{prod}}

\newcommand{\ADComment}[1]{\textcolor{red}{AD: #1}}
\newcommand{\AFComment}[1]{\textcolor{red}{AF: #1}}

\begin{document}

\section*{Section B}


\subsection*{Problem Description: a thread-safe ECO-Logic}

In this section you will create a virtual market place where different agents
can sell and buy plastic products.
Every agent is characterized by an action it will repeatedly perform until
interrupted.


There are four types of agents:
\begin{itemize}
	\item \j{Chemical plant}: produce \emph{new} raw plastic batches
	\item \j{Manufacturer}: buy a certain amount of raw plastic batches and
assemble them into plastic goods to be put on the market
	\item \j{Consumer}: buy plastic goods, use them, and them trash them
	\item \j{Recycle center}: collects trashed goods and recycle their plastic
components producing new \emph{recycled} raw plastic batches, which can be used
to manufacture new goods
\end{itemize}

Your goal is to implement the behavior of each such class of agents, the market
place where all the sell and buy operations happen, and some utility classes.

We will begin with these utility classes and then detail your tasks for 
implementing the agents and the market place.

As usual, for this part of the test you are required to implement your own
(thread-safe) data structure, and you cannot therefore use built-in Java
collections for these tasks.

\section*{Getting Started}

The project containing the skeleton files is located in your Lexis home
`\AFComment{TODO}' directory at:

\begin{itemize}
\item \AFComment{TODO} \j{\textasciitilde/TODO/SectionB/}
\end{itemize}


During the test, you will modify \textbf{only} the following files, among those
provided in the skeleton. You are free to add as many files as you see fit,
but not to modify provided source files not listed below. You can add tests or
otherwise modify the test suite.

\begin{itemize}
	\item \j{utils/\{SafeQueue.java, UnsafeQueue.java\}}: where you will implement
a thread-safe and a non thread-safe queue data structure
	\item \j{domain/agents/\{ChemicalPlant, Manufacturer, RecycleCenter\}}: where
you
will implement the behavior of the different kinds of agents
	\item \j{domain/MarketPlaceImpl.java}: where you will implement the basic
features of the  market place
	\item \j{Main.java}: which you may implement to help you debug your code (not
assessed)
\end{itemize}

Do not change the names or the contents of any other provided source class
(in the folder \j{src}).

\subsection*{Testing}

You are provided with a set of test cases in the \j{test} directory. The tests
aim at exercising a variety of behaviors of your implementation and to further
explain what your code is expected to do. The test suite is not exhaustive:
even if your solution passes all the tests, your work will be assessed by the
examiners, who may also use a different test suite to check your code. 


\section*{What to do}

\begin{enumerate}
	\item \textbf{Queues.}

	In the package \j{utils}, you will find two classes implementing the interface
\j{Queue}. \j{Queue} provides three methods: \j{void push(E)} adds an element
at the end of the queue, \j{Optional<E> pop()} returns the element at the
beginning of the queue or an empty optional if the queue is empty, and \j{int
size()} returns the number of elements in the queue. A queue implements a
First-In First-Out (FIFO) policy, i.e., the first element pushed in the queue
will be the first one returned by a pop operation. 

Your tasks are:
	\begin{itemize}
		\item implement the class \j{utils.UnsafeQueue}. This class is
not required to be thread-safe. It should allow to store an arbitrary number of
elements. You are free to select the internal representation that you see
better fit. \textbf{[10 marks]}
		\item implement the class \j{utils.SafeQueue}, which is required to be
functionally equivalent to \j{utils.UnsafeQueue} (i.e., to implement a queue),
but this time it has to be thread-safe. A basic coarse-grained implementation
sequentializing all the accesses is worth up to 7 marks. Any
correctly implemented finer-grained
access control, regardless of its performance, can be awarder full marks.

\textbf{[15 marks]}
	\end{itemize}

\item \textbf{Market place.}

You are required to implement the class \j{domain.goods.MarketPlaceImpl}. This
class implements three sets of methods (from the interface \j{MarketPlace}.
\j{sellRawPlastic} and \j{buyRawPlastic} allow a producer of raw plastic
batches
(either a chemical plant or a recycle center) to sell them
through the market. Manufacturers can then buy the available batches, if any.
The market implements a FIFO policy with priority: \emph{recycled} plastic
batches have higher priority than \emph{new} plastic batches when the method
\j{buyRawPlastic} is invoked; among \emph{recycled} raw plastic batches the
order in which the batches are available for buying is the same as the order in
which they have been placed on the market for selling. (See also
\j{MarketPlaceTest} for some examples of expected behavior.) If no batches of
raw plastic are available, \j{buyRawPlastic} returns an empty optional. From an
external perspective, the methods \j{sellRawPlastic} and \j{buyRawPlastic}
behaves as the \j{push} and \j{pop} methods of a priority queue, whose elements
are sorted first on the base of their priority (recycled comes before new) and
then FIFO.

The methods \j{sellPlasticGood} and \j{buyPlasticGood} allow a manufacturer to
sell plastic goods on the market and a consumer to buy them, respectively.
Plastic goods do not have priority values and are available for buying in the
same order in which they have been registered fro selling (FIFO).

 The methods
\j{disposePlasticGood} and \j{collectDisposedGood} allow consumers to dispose
of their plastic goods and recycle centers to collect disposed goods,
respectively. Also in this case, disposed goods are available for collection in
the same order as they have been disposed of.

A market place instance must be thread-safe. Multiple agents may invoke any of
its
methods at any time. (If useful, you may reuse (part of) your queues'
implementation.)

\textbf{[15 marks]}

\clearpage

\item \textbf{Agents.}

The different types of agents are described by corresponding classes in the
package \\\j{domain.agents}. All the agents extend the abstract class \j{Agent}
(that you should not modify). \j{Agent} extends \j{Thread} and its \j{run()}
method repeatedly invokes \j{doAction()}, and waits a random time after
each such invocations. The loop can be interrupted invoking the method
\j{interrupt} or if the thread itself is interrupted. The class \j{Agent} also
provides a method \j{think()} that realize the random waiting time, and gives
access to its subclasses the protected field \j{marketPlace}. 
An agent is instantiated specifying the average thinking time (delay between to
consecutive invocations of \j{doAction()}) and the market place in which the
agent is expected to operate.

Each subclass of \j{Agent} is required to override the method \j{doAction()} to
implement the logic of the different types of agents as described below
(again, you should not modify any class in the \j{goods} package):

\begin{itemize}
	\item \j{ChemicalPlant}: a chemical plant agent produces at each invocation of
\j{doAction()} a new \j{RawPlastic} batch and sells it on the market place.
The constructor of \j{RawPlastic} takes one argument specifying if the batch is
from new or recycle material. \j{ChemicalPlant} uses only new material. Only
for this class, the implementation is provided as an example. You are not
required to modify it.
	\item \j{Consumer}: a consumer tries to buy a plastic good from the market. If
such good is available (the optional is not empty), it uses the good for a
certain time (you can invoke the method \j{think()} to simulate such wait), and
then disposes it \\ (\j{MarketPlace.disposePlasticGood()}). 

\textbf{[3 marks]}

	\item \j{Manufacturer}: the signature of the constructor of this class
requires the number of plastic batches needed by the manufacturer to produce
each plastic good (if such argument is less than 1, an
\j{InvalidArgumentException} should be thrown).
At each invocation of its action, a manufacturer will
repeatedly try to buy a raw plastic batch, until it has collected the number
required to produce a new plastic good; if the invocation of
\j{MarketPlace.buyRawPlastic()}
returns an empty optional, the manufacturer waits some time (\j{think()})
before
trying again. When enough raw plastic batches have been collected, the
manufacturer produces a plastic good (passing the collected raw plastic batches
as argument to the constructor of \j{PlasticGood}) and sells it on the market.

\textbf{[3 marks]}

\item \j{RecycleCenter}: every time its action is invoked, a recycle center
tries to collect a disposed good from the market. If such good is available,
the recycle center iterates over the basic
materials composing the good (\j{PlasticGood.getBasicMaterials()}) and recycle
them according to the following rules.
Each batch of raw plastic with origin \emph{new} is used to produce one batch
of \emph{recycled} raw plastic. Every two batches of \emph{recycled} plastic
from disposed goods are used to produce one batch
of \emph{recycled} raw plastic. The produced batches of \emph{recycled} raw
plastic are sold on the market. Notice that it is not required that the two
batches of \emph{recycled} plastic (which are recycled again into one new
batch) 
come from
the same disposed good (and therefore collected during a single invocation of
the method \j{doAction()}). See also the last test case in 
\j{RecycleCenterTest} for more details.
The recycle center only needs to count how many batches
of disposed material it processed and produce the corresponding number of
\emph{recycled} raw plastic batches to be sold on the market (by constructing
new \j{RawPlastic} instances).

\textbf{[4 marks]}
\end{itemize}

\end{enumerate}



\textbf{Total for Section B: 50 marks}

\end{document}
