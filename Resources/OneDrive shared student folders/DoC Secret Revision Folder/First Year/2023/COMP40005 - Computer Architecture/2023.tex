\documentclass[11pt]{article}

% Intra-PDF Links and Bookmarks 
\usepackage[
  % color my links blue
  colorlinks=true,
  linkcolor=black,
  urlcolor=blue,
  % necessary flag for PDF metadata to be written
  luatex,
  % various PDF metadata for SEO and good UX
  pdfauthor={Imperial College London},
  pdftitle={COMP40005 Computer Architecture Exam May 2023},
  pdffitwindow=true,
  pdflang=en-GB,
  pdfmenubar=false,
  pdfpagelayout=OneColumn,
  pdfnewwindow=true
]{hyperref}

% in supported viewers, fit the page to horizontal width
\hypersetup{pdfstartview={FitH top}}

% A4, 0.5cm margins, header+footer
\usepackage[a4paper, margin=1.91cm, top=2.91cm, bottom=2.91cm]{geometry}

% Use UTF-8 so that we can have lots of nice characters!
% uncomment if not using LuaLaTeX
% \usepackage[utf8]{inputenc}
\usepackage[english]{babel}

\title{COMP40005 Computer Architecture Exam 2023}
\author{Imperial College London}
\date{\today}

\begin{document}

\maketitle

\begin{enumerate}
    \item \begin{enumerate}
        \item Is the number of instructions executed per second a reliable way of comparing performance of difference processors? Provide three reasons to support your answer.
        \item A subset of the instructions for processor $P$ can be sped up by $n$ times when executing them on the accelerator $A$. When a program $Q$ is compiled into the instructions of $P$ such that a fraction $f$ belongs to this subset, what is the speed improvement of $P$ when it is accelerated by $A$?
        \item It is found that the accelerator $A$ costs $c$ times as much as the processor $P$. What is the minimum fraction of instructions for programs that $A$ has to speed up such that the accelerated processor is $c$ times faster than $P$?
        \item As technology advances, the performance of $P$ is expected to improve by $m$ times per month. How many months would pass when $P$ alone can execute the program $Q$ as fast as the current $P$ when it is accelerated by $A$?
        \item When a program $R$ is compiled into the instructions of $P$, a fraction $f$ of the instructions can be sped up $n$ times by the accelerator $A$ as before, and additionally a further fraction $g$ of the instructions can be sped up $k$ times by another accleerator $B$. What is the speed improvement of $P$ when it is accelerated by both $A$ and $B$?
    \end{enumerate}

    \vfill
    \emph{The five parts carry, respectively, 15\%, 10\%, 15\%, 25\%, 35\% of the marks.}

    \pagebreak

    \item \begin{enumerate}
        \item Compute the offset of each field, the size of the structure and its alignment requirement for x86-64, per each of the following structure declarations:
        \begin{verbatim}
            struct d1 { short i; long j; int *k; short *l; };
            struct d2 { char a[3]; char *b[4] };
            struct d3 { struct d1 c; struct d2 e[2]; }; \end{verbatim}
        Suppose we have a pointer D3 type of \texttt{struct d3} and the address of \texttt{struct d3} is stored in register \texttt{\%rdi}, and integer index $i$ in \texttt{\%rsi}, respectively. Write the assembly code line for the following pseudo code instructions:
        \begin{enumerate}
            \item Store the value \texttt{D3->c.k} to register result \texttt{\%rax}.
            \item Store the value \texttt{D3->e[index].a} to register result \texttt{\%rax}.
        \end{enumerate}

        \item Consider the following x86-64 optimised assembly function:
        \begin{verbatim}
            sum:
                pushq %rbx
                movq %rdi, %rdx
                movl %esi, %edi
                call randomise
                testl %eax, %eax
                je .L4
                movl %eax, %eax
                leaq (%rbx, %rax, 8), %rdx
                movl $0, %eax
            L3:
                addq (%rbx), %rax
                addq $8, %rbx
                cmpq %rdx, %rbx
                jne .L3
            .L1:
                popq %rbx
                ret
            .L4
                movl $0, %eax
                jmp .L1
        \end{verbatim}

        Fill in the blanks in the following \texttt{sum} function. You may only use the variable names \texttt{result}, \texttt{count}, \texttt{init} and \texttt{seed}, but not the register names.
        \begin{verbatim}
            long sum(long *init, int seed) {
                int result = ___;
                int count = ___;
                while (___) {
                    ___;
                    ___;
                    ___;
                }
                return ___;
            }
        \end{verbatim}

        \pagebreak

        \item A certain byte addressable memory system is 12 bits wide. 
        Memory accesses are to 1-byte words. 
        The cache is 2-way set associated (two lines), has 4 sets and the blocks are 4 bytes in size.
        The contents of the cache are listed below. 

        \begin{center}
            \begin{tabular}{|l|l|l|l|l|l|l||l|l|l|l|l|l|}
                \hline
                Index & Tag & Valid & B0 & B1 & B2 & B3 & Tag & Valid & B0 & B1 & B2 & B3 \\
                \hline 
                0 & 12 & 1 & A2 & DF & E4 & DD & 05 & 1 & 25 & CF & 1D & F8 \\
                1 & C5 & 0 & 25 & 23 & E4 & ED & 47 & 0 & 5A & 09 & 67 & AD \\
                2 & F3 & 0 & 3B & F2 & 43 & 4D & 42 & 1 & DD & 7E & 34 & AB \\
                3 & 97 & 1 & DD & FF & 34 & FC & BB & 0 & 5B & AC & 7B & DD \\
                \hline 
            \end{tabular}
        \end{center}

        The first tag, valid bit and bytes $0-3$ correspond to line 0. The second tag, valid bit and bytes $0-3$ correspond to line 1. 
        
        Provide the format of \texttt{0x42B} and \texttt{0x474} addresses.
        Indicate the fields and bits that would be used to determine the block offset, set index and tag of each address in hexadecimal format. 
        Also, indicate the cache entry for these two addresses, i.e. whether a cache hit or miss occurs in each of the two addresses. 
        If a cache hit occurs, provide the cache byte returned.

        In summary, for \texttt{0x42B} and \texttt{0x474} addresses provide
        \begin{itemize}
            \item the block offset bits, index bits and tag bits,
            \item whether it is a cache hit or a miss,
            \item the cache byte returned if it is a hit.
        \end{itemize}
    \end{enumerate}

    \vfill
    \emph{The two parts carry, respectively, 75\% and 25\% of the marks.}
\end{enumerate}

\end{document}